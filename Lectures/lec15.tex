\section{Student Talk: Classical Simulation of Noisy Circuits}

\subsection{Motivation for Classical Simulation}
Simulating quantum circuits is the basis of many supremacy experiments, and is a leading proposal for demonstrating tasks that exceed classical computational power.
\begin{itemize}
    \item Ideally, circuits rapidly become highly entangled and anti-concentrated, producing output distributions believed to be hard to sample
    \item Noisy case: With realistic constant rate noise, circuits lose structure with depth, raising the possibility that noisy quantum circuits may be easy to simulate.
    \item Depolarizing noise damps Pauli components with large Hamming weight, so only low-weight Pauli paths make non-negligible contributions to the noisy output.
    \item Truncation: Noisy output distribution can be well-approximated by summing over Pauli paths whose weight is at most $l$
    \item Noisy RCS is classical simulable in poly time, so noise eliminates the hardness guarantee - reduces the complexity
\end{itemize}

\subsection{MPS and Noise/Entanglement Scaling}
\begin{itemize}
    \item This concept motivates studying how noise limits the physical Hilbert space complexity, by preventing entanglement in the state from growing without bound.
    \item Classical simulation becomes efficient when the entanglement of the noisy circuits saturates, leading to bounded bond dimension in tensor network/MPS representations.
    \item Sets the stage for practical algorithms.
    \item Exact simulation requires a vector of size $\sim 2^{N}$ - highly entangled parts of $\mathcal{H}$ are hard to simulate. (Note: Bit of a physicist's simplification/intuition - we have to measure in QC, so whether it is completely necessary to store this vector is not obviou - the power of quantum computing is more subtle than this, e.g. entanglement/interference).
    \item If each two-qubit gate has fidelity $f = 1-\e$ then for an $N$-qubit, depth $D$ circuit we have $F \sim (1-\e)^{ND}$ so we can't reach the very highly entangled state. 
    \item We can compress the wavefunction with high accuracy for low entanglement levels. This compression does introduce a finite error rate.
    \item Across a bipartition, we write the Schmidt decomposition:
    \begin{equation}
        \ket{\Psi} = \sum_\mu S_\mu \ket{\mu}_A\ket{\mu}_B
    \end{equation}
    with the significant $S_\mu$ values determining the required bond dimension $\chi$. Entanglement entropy is the Shannon entropy of the singular values.
    \item We can then write an MPS:
    \begin{equation}
        \ket{\Psi} = \sum_{\set{i_n}}M^{(1)i_1}M^{(2)i_2}\ldots M^{(N)i_N}\ket{i_1 \ldots i_N}
    \end{equation}
    \item Noise keeps $\chi$ small because noisy circuits supress singular values
    \item Two qubit updates scale as $O(\chi^3)$, so a bounded $\chi$ enables efficient classical simulation.
    \item Remark: Here we think about density matrices $\rho_\Psi$ after noise, so we actually think about the MPS of the purification.
    \item Key finding - as the depth of the circuit increases, the fidelity:
    \begin{equation}
        f_n = \abs{\bra{\Psi_T(n)}C_z\ket{\Psi_T(n)}}^2 \to f_\infty
    \end{equation}
    a depth independent value.
    \item Remark: Noise is assumed to be i.i.d.
    \item For random circuits, successive truncation errors are multiplicative. The quantum noisy circuit can be estimated when:
    \begin{equation}
        F \approx (f_\infty(\chi))^{ND/2}
    \end{equation}
    with the entire simulation governed by the per-gate fidelity $f_\infty(\chi)$.
    \item Large $\chi$ reduces $\e_\infty(\chi) = 1 - f_\infty(\chi)$, but after a point, improving accuracy requires exponentially large $\chi$. Impies a finite saturated $\e_\infty(\chi)$ bounds the MPS bond dim.
\end{itemize}

\subsection{Application to 2D circuit + Conclusions/Implications}
\begin{itemize}
    \item 2D circuits hard to map to MPS
    \item Block grouping approach of qubits - keeps $\chi$ manageable.
    \item Depth-20 circuit on 54 occupies only $1.5 \times 10^{-6}$\% of the Hilbert space. I.e. close to the maximally mixed state (product state).
    \item Conclusion: Noise restricts state complexity for classical TN methods to be accessible.
    \item Implications for ``quantum advantage''; dependence of classical hardness of RCS and its dependence on noise, other messages of beyond-the-noise assumptions made in this paper etc.
\end{itemize}