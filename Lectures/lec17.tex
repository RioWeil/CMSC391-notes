\section{Efficient Simulation of Gaussian Boson Sampling}
\subsection{Setup}
\begin{itemize}
    \item Inputs are squeezed states + vacuum states rather than fock states + vacuum states.
    \item Exact simulation of probabilities is $\#P$-hard
\end{itemize}

Amounts to computing a halfnian:
\begin{equation}
    \text{haf}(A) = \sum_{\rho \in P_n}\prod_{\set{i, j} \in \rho}A_{i, j}
\end{equation}
sums over all perfect pairings of indices, and encodes pair-creation correlations. E.g. for $A \in \CC^{4\times 4}$:
\begin{equation}
    \text{haf}(A) = A_{12}A_{34} + A_{13}A_{24} + A_{14}A_{23}
\end{equation}

Note: Permanent is perfect matching on a bipartite graph, Halfnian is perfect matching on an arbitrary graph.

\subsection{Methods}

What we do instead - look for low mode marginals:
\begin{itemize}
    \item Single mode: Probability of a single mode's detector outcome, i.e. clicks or not.
    \item Two-mode: Joint distribution over pairs of modes.
\end{itemize}
These can be computed in polynomial time.

Method 1: Boltzmann machines

\begin{equation}
    \underbrace{p(\v{z})}_{\text{prob. of $\v{z}$}} = \frac{1}{Z}\exp(\sum_a \underbrace{\lambda_a}_{\text{bias}} z_a + \sum_{a < b}\underbrace{\lambda_{ab}}_{pairwaise correlations}z_az_b + \sum_{a < b < c}\lambda_{abc}z_az_bz_c)
\end{equation}

Training of BM model: Gradient descent to optimize BM parameters such that its 1-2 mode marginals match the ideal GBS marginals. Use Thouless/Anderson/Palmer mean field.

Method 2: Greedy heuristic
\begin{enumerate}
    \item Initialize with all 0s
    \item Iterate (flip that improves agreement)
    \item Evaluate improvement
    \item Repeat until best approximation is found to target marginals.
\end{enumerate}

We can look at statistical distance between two probability distributions in 2 ways; Total variation distance (TVD):
\begin{equation}
    \delta = \frac{1}{2}\sum_z \abs{p(z) - q(z)}
\end{equation}
and Kullback-Leibler divergence:
\begin{equation}
    D_{KL}(p, q) = \text{XE}(p, q) - H(p) = \sum_\v{z} p(\v{z})\log\frac{p(\v{z})}{q(\v{z})}
\end{equation}

\subsection{Results + Limitiations}
Looked at $\leq 14$ mode marginals, which are tractable to compute. They find that their techniques outperforms thermal samplers (i.e. experiment with thermal noise).

Does not apply to RCS because there the marginals are essentially uniform (i.e. no information in the marginals). Also, the classical sampler does not recover full global bit-string probabilities.