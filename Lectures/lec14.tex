\section{Student Talk: The Complexity = Volume conjecture}

\subsection{Introducing the Conjecture}
The AdS/CFT conjecture states that we can map a $d$-dimensional anti-de-Sitter spacetime to a $d-1$ conformal field theory on the boundary. For the purposes of this lecture, it is the statement that:
\begin{equation}
    \set{\text{black hole}} \cong \set{\text{thermal state}}
\end{equation}

Let us take $\mathcal{H}_a \cong \mathcal{H}_B \cong \CC^N$, and consider the computational basis state $\set{\ket{i}}_i$. The thermal double state of the black hole is the maximally entangled state:
\begin{equation}
    \ket{\Phi} = \frac{1}{\sqrt{N}}\sum_i \ket{i}_A \otimes \ket{i}_B
\end{equation}

We recall the bounceback property of maximally entangled states:
\begin{equation}
    \frac{1}{\sqrt{N}}\sum_i V\ket{i}_A \otimes \ket{i}_B = \frac{1}{\sqrt{N}}\sum_i \ket{i}_A \otimes V^*\ket{i}_B
\end{equation}
Therefore:
\begin{equation}
    \frac{1}{\sqrt{N}}\sum_i V[\ket{i}_A \otimes \ket{i}_B] = \frac{1}{\sqrt{N}}\sum_i \ket{i}_A \otimes U\ket{i}_B
\end{equation}
with $U = (V^*)^2$, thus the black hole evolves unitarily and is efficiently implementable.

Let us return to thermal properties of the black hole. Consider local unitaries $O_A$ on $\mathcal{H}_A$ and Heisenberg evolved $U(t)O_BU^*(t)$ on $\mathcal{H}_B$. Then:
\begin{equation}
    \bra{\psi_t}O_A \otimes U(t)O_BU^*(t)\ket{\psi_t} = \bra{O_A^*, U(t)O_BU^*(t)}_{\text{HS}}
\end{equation}
The Hilbert-Schmidt norm is only zero if the two operators have disjoint support. If we let $U(t) = e^{-iH_{\text{CFT}}t}$. Since $H_{\text{CFT}}$ is scrambling, then by time $t \approx t_{\text{scr}} \sim \frac{\beta}{2\pi}\log N$:
\begin{equation}
    \avg{O_A^*, UO_BU^*}_{\text{HS}} \to 0
\end{equation}
exponentially quickly. This is because $O_B$ delocalizes via the scrambling.

Now, we introduce the complexity = volume conjecture. The volume of the black hole is approximately linearly increasing. If thermalization happens in the CTT, what is the CFT dual of the volume? 

The conjecture is that the CFT dual of the black hole volume is the quantum circuit complexity of $\ket{\psi_t}, \mathcal{C}_\e(\ket{\psi_t})$. In other words:
\begin{equation}
    \mathcal{C}_\e(\ket{\psi_t}) \approx cV(\ket{\Phi_t})
\end{equation}
where $\mathcal{C}_\e$ is the circuit complexity (in terms of gate count starting from a product state) of preparing $\ket{\psi_t}$, up to error epsilon $\e$ (so the complexity is not necessarily coming from the fact that exact states may be hard to prepare).

\subsection{Constraints on AdS/CFT}
First, let us define the notion of \emph{pseudorandomness}. The ensemble $\set{\ket{\psi_k}}_k$ is pseudorandom if no polynomial time quantum computation can differentiate between $\ket{\psi_k}$ and a Haar random state $\ket{\phi}$.  Note that this implies that $\delta(\ket{\psi_k}, \ket{\phi}) < \e$ in statistical distance.

We begin with a state $\ket{\psi}$, and alternately apply a unitary $U$ adn a random Pauli for $l$ rounds. We then have:
\begin{equation}
    \ket{\Psi_k} = k_l U k_{l-1} U \ldots k_1 U\ket{\psi}, k_j \in \set{I, X, Y, Z}
\end{equation}
then $\ket{\Psi_k} \in D_{U, l,\ket{\phi}}$.

Now, we assume:
\begin{itemize}
    \item $\ket{\psi_k}$ is pseudorandom with $U = e^{-iH_{\text{CFT}}mt_{\text{scr}}}$
    \item We assume that while black holes aren't measurable, they can be at least approximated.
\end{itemize}

Why do we need the ``keys'' in this construction? If we have the ability to apply $U^*$, then $U\ket{0^n}$ and $\ket{\phi}$ Haar random can be easily distinguished by applying $U^*$ and then measuring in the computational basis.

Consider two pseudorandom states $\ket{\psi}, \ket{\psi'}$ on volumes $O(N^2), O(N^3)$ (different evolution times for different $k, k'$ so different wormhole volumes). Let $\Theta$ is the dictionary map $\ket{\Phi} \stackrel{\Theta}{\to}\ket{\psi}$. Because the volumes are distinct enough, we can measure their difference in polynomial time.

Seeming violation of quantum extended church-turing thesis (every efficiently solvable problem can be simulated on a quantum computer) - $\ket{\psi}, \ket{\psi'}$ can be distinguished in poly time in ADS, but a QC in by assumption cannot distinguish the two CFT states (because if both are individually indistinguishable from Haar random states, then they are mutually indistinguishable)!

So either:
\begin{itemize}
    \item The AdS/CFT correspondence dictionary is exponentially complicated - this may mean that it is not a useful duality...
    \item Or, quantum gravity is able to solve a problem that is exponentially hard for a quantum computer! We need to adapt the extended church turing thesis for quantum gravity.
\end{itemize}

Formally for the first point: the quantum extended church turing thesis does hold, then we must restrict the dictionary map $\Theta$ - if $t(N)$ is the time it takes to ``feel'' the black hole, then:
\begin{equation}
    \mathcal{C}_\e(\Phi)t(N) \geq \exp(N)
\end{equation}

Remark by Bill: It turns out that the argument doesn't actually require the volume = complexity conjecture! Leonard Susskind's argument isn't quite right.