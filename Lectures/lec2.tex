\section{Basics Quantum Mechanics}
\subsection{Qubits}
The simplest quantum system is the ``qubit'', or quantum bit. It is mathematically, a complex unit vector in the vector space $\CC^2$. For our finite-dimensional setting, we can consider the Hilbert space as a complex vector space with an inner product. The states on the space look like:
\begin{equation}\label{eq:qubit}
    \ket{\psi} = \m{\alpha \\ \beta}
\end{equation} 
where $\abs{\alpha}^2 + \abs{\beta}^2 = 1$ for $\alpha, \beta \in \CC$. The qubit is thus the generalization of the classical $0, 1$ bit states of
\begin{equation}
    \ket{0} = \m{1 \\ 0}, \quad \ket{1} = \m{0 \\ 1}
\end{equation}
Notice these are orthogonal vectors. What Eq. \eqref{eq:qubit} says is then that $\ket{\psi}$ is a superposition of $\ket{0}, \ket{1}$:
\begin{equation}
    \ket{\psi} = \alpha\ket{0} + \beta\ket{1}
\end{equation}
where $\alpha, \beta$ are amplitudes.

In Dirac's notation, $\ket{\psi}$ is a ``ket'', representing a column vector. Its Hermitian conjugate is the ``bra'', representing a row vector:
\begin{equation}
    \bra{\psi} = \m{\alpha^* & \beta^*} = \alpha^*\bra{0} + \beta^*\bra{1}
\end{equation}
The inner product between $\ket{\psi} = \alpha\ket{0} + \beta\ket{1}$ and $\ket{\psi'} = \alpha'\bra{0} + \beta'\bra{1}$ is given by:
\begin{equation}
    \braket{\psi}{\psi'} = (\alpha'^*\bra{0} + \beta'^*\bra{1})(\alpha^*\bra{0} + \beta^*\bra{1}) = \alpha'^*\alpha + \beta'^*\beta
\end{equation}
where we have used the orthonormality of the $\ket{0}, \ket{1}$ states.

\subsection{Multi-qubit systems and Entanglement}
A two-qubit quantum state is a unit vector in $\CC^2 \otimes \CC^2$, i.e.:
\begin{equation}
    \ket{\psi} = \sum_{i,j=1}^{2}\alpha_{ij}\ket{i}\otimes\ket{j}
\end{equation}
where $\sum_{ij} \abs{\alpha_{ij}}^2 = 1$. Notice that in general, a two-qubit state $\ket{\psi}$ cannot be written as a tensor product/separable state of the form $\ket{\phi} \otimes \ket{\gamma}$, for one-qubit states $\ket{\phi}, \ket{\gamma} \in \CC^2$. As an example of an entangled (not separable) state, we have the Bell/EPR pair:
\begin{equation}
    \ket{B_{00}} = \frac{1}{\sqrt{2}}\ket{0}_A\ket{0}_B + \frac{1}{\sqrt{2}}\ket{1}_A\ket{1}_B
\end{equation}
and as an example of a separable state we have:
\begin{equation}
    \frac{1}{\sqrt{2}}(\ket{0}_A + \ket{1}_A)\otimes  \frac{1}{\sqrt{2}}(\ket{0}_B + \ket{1}_B) = \frac{1}{2}\left(\ket{0}_A\ket{0}_B + \ket{0}_A\ket{1}_B + \ket{1}_A\ket{0}_B + \ket{1}_A\ket{1}_B\right)
\end{equation}

More generally, the state of an $n$ qubit system is a unit vector in $(\CC^2)^{\otimes n} \cong \CC^{2^n}$. Just as in the two-qubit case, we can divide the $n$ qubits into two registers $A/B$ of size $n_1, n_2$, where the state of register $A$ is in the Hilbert space $\H_A = \CC^{2^{n_1}}$ and the state of register $B$ lives in the Hilert space $\H_B = \CC^{2^{n_2}}$. The state of a bipartite system is a unit vector in $\H_A \otimes \H_B$ of dimension $2^{n_1 + n_2}$ that may be entangled. There are generalizations to tripartite systems and beyond, though notions of entanglement are more subtle there.