\section{Partial Measurement and Interference}

\subsection{Partial Measurement}
Let us consider quantum measurement again. Suppose we now have a bipartite system with composite Hilbert space $\H_A \otimes \H_B$. What happens if we only measure one subsystem? Supposing we have the bipartite state:
\begin{equation}
    \ket{\psi} = \sum_{ij}c_{ij}\ket{i}_A\ket{j}_B = \sum_j(\sum_i c_{ij}\ket{i}_A)\otimes \ket{j}_B
\end{equation}
if we measure system $B$ in the standard basis, we obtain $\ket{j}_B$ with probability $\sum_i \abs{c_{ij}}^2$, with the combined system collapsing to the state $(\sum_i c_{ij}\ket{i}_A) \otimes \ket{j}_B$ (up to normalization).

As an example, suppose we have $f: \set{0, 1}^n \to \set{0, 1}$ a Boolean function. Consider now the bipartite state $\frac{1}{\sqrt{2^n}}\sum_x \ket{x}_A\ket{f(x)}_B$. 

If we measure register $A$ (in the computational basis), we just have a uniform probability of measuring any bitstring $\ket{x}$, and then register $B$ is left in the state $\ket{f(x)}$ for the measured $x$.

If we instead measure register $B$ (in the computational basis), we obtain outcome $\ket{0}$ with probability proportional to the number of inputs $x$ such that $f(x) = 0$ and the same for outcoem $\ket{1}$, i.e. $\ket{0}$ occurs with probbaility $\sum_{x \in f^{-1}(0)}\left(\frac{1}{\sqrt{2^n}}\right)^2 = \frac{\abs{f^{-1}}(0)}{2^n}$ and $\ket{1}$ with $\frac{\abs{f^{-1}}(0)}{2^n}$. For outcome $b \in \set{0, 1}$ we have the post-measurement state:
\begin{equation}
    \frac{1}{\sqrt{f^{-1}}(b)}\sum_{x \in f^{-1}(b)}\ket{x}\ket{b}
\end{equation}

\subsection{Interference}
Quantum computers seem to be dramatically faster than their classical counterparts at certain tasks, but where does this power come from? The best answer appears to be a phenomenon known as quantum interference. 

Consider the one-qubit state $\ket{\psi} = \ket{0}$ and let $U = \frac{1}{\sqrt{2}}\m{1 & -1 \\ 1 & 1}$. If we apply $U$ to $\ket{\psi}$, we have:
\begin{equation}
    U\ket{\psi} = \frac{1}{\sqrt{2}}\ket{0} + \frac{1}{\sqrt{2}}\ket{1}.
\end{equation}
If we apply $U$ again, we have:
\begin{equation}
    U\left(\frac{1}{\sqrt{2}}\ket{0} + \frac{1}{\sqrt{2}}\ket{1}\right) = \frac{1}{2}\ket{0} + \frac{1}{2}\ket{1} - \frac{1}{2}\ket{0} + \frac{1}{2}\ket{1} = \ket{1}
\end{equation}
so the $\ket{0}$ states cancel out! The outcome of a measurement in the computational basis would now be deterministic. This is one example of destructive interference.

The above example is simple, but tells us that the negative amplitudes are important! In a loose sense, different quantum computational paths cancel due to these relative phases. this is in contrast to classical computation, where randomized pathways always add - randomized algorithms are just deterministic algorithms with random seeds, i.e. a deterministic algorithm $B$ with output probability:
\begin{equation}
    \text{Pr}_{r \in \set{0, 1}^q}[B(x, r) = 1] = \sum_{r: B(x, r) = 1}\text{Pr}[r].
\end{equation}
This is a sum of all positive terms. Note that there is no way to generically express the output probability of a quantum algorithm in this form (we will return to this later!)